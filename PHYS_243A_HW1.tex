
% !TEX TS-program = pdflatex
% !TEX encoding = UTF-8 Unicode


%%%%%%%%%%%%%%%%%%%%%%%%%%%%%%%%%%%%%%%%%%%%%%%%%%%%%%%%%%%%%%%%%%%%%%
%%%%%%%%%%%%%%%%%%%%%%%%%%%%%%	HEADER %%%%%%%%%%%%%%%%%%%%%%%%%%%%%%%%%%%%
\documentclass[12pt]{article} 
\usepackage{amsmath, amsthm, amssymb} % AMS Math Package
\usepackage{graphicx} % Allows for eps images
\usepackage{multicol} % Allows for multiple columns
\usepackage{framed,color}
\usepackage{fancybox}
\usepackage{mdframed}
\usepackage{float}
\usepackage{caption,subcaption}

\usepackage{geometry}
\geometry{
top=0.5in,
inner=0.5in,
outer=0.5in,
bottom=0.5in,
headheight=1ex,
headsep=3ex
}


 % Sets margins and page size
\pagestyle{empty} % Removes page numbers
\makeatletter % Need for anything that contains an @ command 
\renewcommand{\maketitle} % Redefine maketitle to conserve space
{ \begingroup \vskip 10pt \begin{center} \large {\bf \@title}
	\vskip 10pt \large \@author \hskip 20pt \@date \end{center}
  \vskip 10pt \endgroup \setcounter{footnote}{0} }
\makeatother % End of region containing @ commands
\renewcommand{\labelenumi}{(\alph{enumi})} % Use letters for enumerate
% \DeclareMathOperator{\Sample}{Sample}
\let\vaccent=\v % rename builtin command \v{} to \vaccent{}
\renewcommand{\v}[1]{\ensuremath{\mathbf{#1}}} % for vectors
\newcommand{\gv}[1]{\ensuremath{\mbox{\boldmath$ #1 $}}} 
% for vectors of Greek letters
\newcommand{\uv}[1]{\ensuremath{\mathbf{\hat{#1}}}} % for unit vector
\newcommand{\abs}[1]{\left| #1 \right|} % for absolute value
\newcommand{\avg}[1]{\left< #1 \right>} % for average
\let\underdot=\d % rename builtin command \d{} to \underdot{}
\renewcommand{\d}[2]{\frac{d #1}{d #2}} % for derivatives
\newcommand{\dd}[2]{\frac{d^2 #1}{d #2^2}} % for double derivatives
\newcommand{\pd}[2]{\frac{\partial #1}{\partial #2}} 
% for partial derivatives
\newcommand{\pdd}[2]{\frac{\partial^2 #1}{\partial #2^2}} 
% for double partial derivatives
\newcommand{\pdc}[3]{\left( \frac{\partial #1}{\partial #2}
 \right)_{#3}} % for thermodynamic partial derivatives
\newcommand{\ket}[1]{\left| #1 \right>} % for Dirac bras
\newcommand{\bra}[1]{\left< #1 \right|} % for Dirac kets
\newcommand{\braket}[2]{\left< #1 \vphantom{#2} \right|
 \left. #2 \vphantom{#1} \right>} % for Dirac brackets
\newcommand{\matrixel}[3]{\left< #1 \vphantom{#2#3} \right|
 #2 \left| #3 \vphantom{#1#2} \right>} % for Dirac matrix elements
\newcommand{\grad}[1]{\gv{\nabla} #1} % for gradient
\let\divsymb=\div % rename builtin command \div to \divsymb
\renewcommand{\div}[1]{\gv{\nabla} \cdot #1} % for divergence
\newcommand{\curl}[1]{\gv{\nabla} \times #1} % for curl
\let\baraccent=\= % rename builtin command \= to \baraccent
\renewcommand{\=}[1]{\stackrel{#1}{=}} % for putting numbers above =
\newtheorem{prop}{Proposition}
\newtheorem{thm}{Theorem}[section]
\newtheorem{lem}[thm]{Lemma}
\theoremstyle{definition}
\newtheorem{dfn}{Definition}
\theoremstyle{remark}
\newtheorem*{rmk}{Remark}

\usepackage[utf8]{inputenc} % set input encoding (not needed with XeLaTeX)
\usepackage{tikz} %for making pretty figures
\usepackage[tikz]{bclogo}
\usepackage{tikz-3dplot}
\usetikzlibrary{calc}

%%% HEADERS & FOOTERS
\usepackage{fancyhdr}
\setlength{\headheight}{15pt}
\pagestyle{fancyplain}
\usepackage{lastpage}
\usepackage{siunitx}
\pagenumbering{arabic}

\lhead{Kyle Hoke}
\chead{PHYS 243A, Homework 1\rightmark}
\rhead{Page \thepage\ / \pageref{LastPage}}
\renewcommand{\headrulewidth}{0.4pt}
\fancyfoot[C]{}

%%% section* TITLE APPEARANCE
\usepackage{sectsty}
\allsectionsfont{\sffamily\mdseries\upshape} % (See the fntguide.pdf for font help)

%%% User Defined Commands %%%%
\newcommand{\f}[1]{f_{#1}(x)}
\newcommand{\axis}{
		\draw[->,thick](0,0) -- (0,5);
		\node[above] at (0,5) {$x$};
		\draw[->,thick](0,0) -- (5,0);
		\node[right] at(5,0){$y$};
		}

%%%%%%%%%%%%%%%%%%%%%%%%%%%%%%%%%%%%%%%%%%%%%%%%%%%%%%%%%%%%%%%%%%%%%%%%%%%%%%%
%%%%%%%%%%%%%%%%%%%%%%%%%%%%%%%%%%%%%% END OF HEADER %%%%%%%%%%%%%%%%%%%%%%%%%%%%%%%%%
%%%%%%%%%%%%%%%%%%%%%%%%%%%%%%%%%%%%%%%%%%%%%%%%%%%%%%%%%%%%%%%%%%%%%%%%%%%%%%%





%%% BEGIN DOCUMENT %%%%%%%%%%%%%%%%%%%%%%%%%%%%%%%%%%%%%%%%%%%%%%%%%%%%%%%%%%%%%%%%%%%
\title{\LARGE PHYS 243A Homework 1}
\author{Kyle Hoke}
%\date{} % Activate to display a given date or no date (if empty),
         % otherwise the current date is printed 

\begin{document}
\maketitle
\thispagestyle{empty}





%%%%%%%%%%%%%%%%%%%%%%%%%%%%%%%%%%%%%%%%%%%%%%%%%%%%%%%%%%%%%%%%%%%%%%%%%%%%%%%%%%%%
%%%%%%%%%%%%%%% PROBLEM 1 %%%%%%%%%%%%%%%%%%%%%
%%%%%%%%%%%%%%%%%%%%%%%%%%%%%%%%%%%%%%%%%%%%%%%%%%%%%%%%%%%%%%%%%%%%%%%%%%%%%%%%%%%%

\section*{1.1}

\begin{bclogo}[logo=\bcquestion , barre=none]
\newline
If we wish to produce a thin-film magnetic storage device with 100 Gbits/in$^2$, each bit is to be 5
times as big in one dimension as the other, the total amount of open area between bits is to be the
same as the total area occupied by the bits, and the film thickness of the medium storing the
information is to be 10 nm, how many atoms are involved in each bit? Assume for simplicity that the
film is pure Co, with a density of 9.0 x 1022 atoms/cm$^3$.
\end{bclogo}
\vspace{1cm}




\begin{figure}[H]
 \centering
 \begin{tikzpicture}
        \draw[-, blue, ultra thick] (0,0,0) -- (-1,1,1);
\end{tikzpicture}
\caption{Single Bit}
\label{escher2}
\end{figure}


%%%%%%%%%%%%%%%%%%%%%%%%%%%%%%%%%%%%%%%%%%%%%%%%%%%%%%%%%%%%%%%%%%%%%%%%%%%%%%%%%%%%
%%%%%%%%%%%%%%% PROBLEM 2 %%%%%%%%%%%%%%%%%%%%%
%%%%%%%%%%%%%%%%%%%%%%%%%%%%%%%%%%%%%%%%%%%%%%%%%%%%%%%%%%%%%%%%%%%%%%%%%%%%%%%%%%%%
\newpage
\section*{1.2}

\begin{bclogo}[logo=\bcquestion , barre=none]
\newline
(a) Begin with the Maxwell-Boltzmann distribution for molecular velocities in an ideal gas as
expressed in x,y,z coordinates, and derive by integration the formula for the rate at which molecules
strike a flat surface of unit area perpendicular to one of the axes. From this, determine the time
necessary to form a monolayer of gas on a surface, assuming a general sticking probability of PS. Note
that this is the same derivation that can be found in many physical chemistry texts to predict the rate at
which a gas leaks out through an orifice.
\newline
(b) Now make the assumption that only the bare surface area remaining on the surface after a given
exposure time is active for a particular gas adsorption, and that PS = unity on the bare area, but zero
elsewhere. Derive the general form of PS as a function of time for this case.\newline
(c) If a surface is exposed to 10$^{-9}$ Torr of CO at ambient temperature, how long will it take to form the first monolayer:
\newline 
(i) If Ps = unity?
\newline 
(ii) If Ps follows the relationship of part (b)?
\end{bclogo}
\vspace{2cm}

We begin with the Maxwell-Boltzmann distribution for molecules in an ideal gas.
\[
D(v) = \sqrt{\left(\dfrac{m}{2\pi kt}\right)^3}4\pi v^2 \exp{\left(-\dfrac{mv^2}{2kt}\right)}
\]

Where $D$ is the probability of finding a molecule at speed $v$. The speed $v^2 = v_x^2 + v_y^2 + v_z^2$ is in three dimensions. We first want to find the average speed of the molecules, then the flux through a single side of a unit cube when the molecules have a given number density. To find an average speed we add all the speeds up weight by their probability.
\[
	\bar{v}  = \sum v D(v) \Delta v,
\]
where $\Delta v$ represents a discrete spacing of speeds. In reality we can have continuum of speeds so to find the average speed we must sum over all possible speeds, which means an integral.

\begin{align*}
	\bar{v} &= \int_{0}^{\infty} v D(v) dv \\[3mm]
		&= \int_{0}^{\infty} \sqrt{\left(\dfrac{m}{2\pi kt}\right)^3}4\pi v^3 \exp{\left(-\dfrac{mv^2}{2kt}\right)} \\[3mm]
		&= \left(\dfrac{m}{2\pi kt}\right)^{3/2} 4\pi \int_{0}^{\infty} v^3 \exp{\left(-\dfrac{mv^2}{2kt}\right)}
\end{align*}
	
Now we make a substitution to make the integration easier: $$x = mv^2/2kt$$. Finding dx/dv and $v^2$ in terms of $x$,  we make the substitutions and get,

\[
	\bar{v} = \left(\dfrac{m}{2\pi kt}\right)^{3/2} 4\pi\left(\dfrac{kt}{m}\right) \left(\dfrac{2kt}{m}\right) \int_{0}^{\infty} x \exp{\left(-x\right)}
\]

Looking up this solution in a table (I used Schaums) we find that the integral evaluates to unity and we end up with.

\[
	\bar{v} =  \left(\dfrac{m}{2\pi kt}\right)^{3/2} 4\pi\left(\dfrac{kt}{m}\right) \left(\dfrac{2kt}{m}\right)
\]






%%%%%%%%%%%%%%%%%%%%%%%%%%%%%%%%%%%%%%%%%%%%%%%%%%%%%%%%%%%%%%%%%%%%%%%%%%%%%%%%%%%%
%%%%%%%%%%%%%%% PROBLEM 3 %%%%%%%%%%%%%%%%%%%%%
%%%%%%%%%%%%%%%%%%%%%%%%%%%%%%%%%%%%%%%%%%%%%%%%%%%%%%%%%%%%%%%%%%%%%%%%%%%%%%%%%%%%


\newpage
\section*{1.3}
\begin{bclogo}[logo=\bcquestion , barre=none]
\newline
 (a) What would be the minimum energy required to take a cube of Pt metal 1.0 cm on a
 side at room temperature and disperse it into tiny cubic "nanoparticles" of 10-6
 cm on a side? Assume
 that this is done in a perfect ultrahigh vacuum environment, with the surfaces in equilibrium with the
 very low vapor pressure of Pt, and make use of the argument in Zangwill, p. 12, but with a key equation
 corrected to
 \newline 
 (b) Estimate the fraction of the atoms that are on the surface of the 1 cm cube. Of the 10-6
 cm cube.
 Assume that the density of Pt atoms is 6.62 x 1022 cm-3. 
\end{bclogo}
\vspace{2cm}







\end{document}
